\documentclass{book}
\usepackage{sstdef}
\title{signal}
\begin{document}

\section{The \cmd{sigpause} program}

\subsection{Usage}
\begin{code}%
  sigpause [ -cdbu \var{sig} -an ] [ \var{prog} ]
\end{code}
where \carg{prog} is one or more arguments specifying a program to run.

\subsection{Description}
Each option that takes a signal as its argument accepts either a number or one
of the (case insensitive) strings \cmd{ALRM}, \cmd{CHLD}, \cmd{CONT}, \cmd{HUP},
\cmd{INT}, \cmd{PIPE}, or \cmd{TERM}. 

Options are:
\begin{itemize}
\item \cmd{-c \var{sig}}: Catch \carg{sig} and unblock during pause.
\item \cmd{-d \var{sig}}: Set default action for \carg{sig} and unblock during pause.
\item \cmd{-b \var{sig}}: Block \carg{sig} during pause.
\item \cmd{-u \var{sig}}: Unblock \carg{sig} during pause.
\item \cmd{-a \var{sig}}: Block all signals during pause.
\item \cmd{-n \var{sig}}: Block no signals during pause.
\end{itemize}

After processing options, \cmd{sigpause} pauses waiting for unblocked signals.
A caught signal causes \cmd{sigpause} to execute \carg{prog}, or exit~0 in the
absence of \carg{prog}.
\end{document}


