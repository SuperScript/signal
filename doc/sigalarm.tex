\documentclass{book}
\usepackage{sstdef}
\title{signal}
\begin{document}

\section{The \cmd{sigalarm} program}

\subsection{Usage}
\begin{code}%
  sigalarm [ -a -s \var{sig} ] \var{timeout} \var{prog}
\end{code}
where \cmd{timeout} is a number of seconds to wait before delivering a signal
and \carg{prog} is one or more arguments specifying a program to run.

\subsection{Description}
The \cmd{-s} option takes a signal as an argument and accepts a number or one of
the (case insensitive) strings \cmd{ALRM}, \cmd{CHLD}, \cmd{CONT}, \cmd{HUP},
\cmd{INT}, \cmd{PIPE}, or \cmd{TERM}.  The last \cmd{-s} or \cmd{-a} option takes
precedence.

The \cmd{sigalarm} sets an alarm for \carg{timeout} seconds.  Options are:
\begin{itemize}
\item \cmd{-a \var{sig}}: Set alarm and execute \carg{prog}.
\item \cmd{-s \var{sig}}: Fork and exec \carg{prog}.  Deliver \carg{sig} to \carg{prog}
      if the alarm goes off.
\end{itemize}
The following invokations are equivalent:
\begin{code}%
  sigalarm \var{prog}
  sigalarm \var{opts} -a \var{prog}
\end{code}

Under \cmd{-a}, \cmd{sigalarm} sets an alarm for \carg{timeout} seconds and
executes \carg{prog}.  The child process inherits the alarm.  Note that
implementations of \cmd{sleep} use the \cmd{alarm} may interact poorly with
\cmd{sigalarm -a}.  With \cmd{-s}, \cmd{sigalarm} forks and execs \carg{prog}, and
delivers it a \carg{sig} signal if \carg{timeout} seconds elapse before the child
exits.  In either case, \cmd{sigalarm} exits with the same exit code as
\carg{prog}.

\end{document}


